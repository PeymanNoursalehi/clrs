\documentclass[a4paper,12pt]{article}
\usepackage{algorithmic}
\newcommand{\newpar}[1]
{\bigskip \noindent \textbf{Exercises #1} \newline}
\newcommand{\newprob}[1]
{\bigskip \noindent \textbf{Problem #1} \newline}
\newcommand{\subpar}[1]{\medskip \noindent #1.}
\newcommand{\la}{\leftarrow}
\newcommand{\ra}{\rightarrow}
\newcommand{\exchange}[2]{\mathrm{exchange}\ #1 \leftrightarrow #2}
\newenvironment{alg}[2]
               {\noindent $\textsc{#1}(#2)$ \begin{algorithmic}}
               {\end{algorithmic}}

\begin{document}

\newpar{8.2-2} From line $10$, we see that if there are $m$ integers
equal to $i$, then they occupy the indexes $C[i], C[i]-1, \ldots,
C[i]-m+1$ in the array $B$.  Since we place the integers from the end
of the array $A$, we then deduce that they keep the order in which
they appears.  Thus \textsc{counting-sort} is a stable sort.

\newpar{8.2-3} Applying the same reasoning as in \textbf{Exercises
  8.2-3} we deduce that if there are $m$ integers equal to $i$, then
they occupy the range $C[i]-m+1, \ldots C[i]$ in $B$ but in reverse
order of their appearances.  Thus the algorithm still works properly
but it's not a stable sort anymore.

\newpar{8.2-4}
\begin{alg}{preprocess}{A, k}
  \STATE \COMMENT{Create an array of size $k+1$ and bind it to the
    global variable $C$}

  \FOR{$i \la 0$ \textbf{to} $k$}
  \STATE $C[i] \la 0$
  \ENDFOR

  \FOR{$i \la 1$ \textbf{to} $\mathrm{length}[A]$}
  \STATE $C[A[i]] \la C[A[i]] + 1$
  \ENDFOR

  \FOR{$i \la 1$ \textbf{to} $k$}
  \STATE $C[i] \la C[i] + C[i-1]$
  \ENDFOR
\end{alg}
\begin{alg}{count-range}{a, b}
  \RETURN{$C[b] - C[a] + 1$}
\end{alg}

\end{document}
