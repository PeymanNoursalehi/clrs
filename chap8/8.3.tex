\documentclass[a4paper,12pt]{article}
\usepackage{algorithmic}
\newcommand{\newpar}[1]
{\bigskip \noindent \textbf{Exercises #1} \newline}
\newcommand{\newprob}[1]
{\bigskip \noindent \textbf{Problem #1} \newline}
\newcommand{\subpar}[1]{\medskip \noindent #1.}
\newcommand{\la}{\leftarrow}
\newcommand{\ra}{\rightarrow}
\newcommand{\exchange}[2]{\mathrm{exchange}\ #1 \leftrightarrow #2}
\newenvironment{alg}[2]
               {\noindent $\textsc{#1}(#2)$ \begin{algorithmic}}
               {\end{algorithmic}}

\begin{document}

\newpar{8.3-2} \textsc{insertion-sort} and \textsc{mege-sort} are
stable.

To make any sorting algorithm stable, first we scan the array and keep
track of the indexes of the keys equal to a given key in the
order of their apparition.  We do this for all the possible values of
the keys.  Then we build an array composed of the first element in
each group and sort it using the sorting algorithm.  Then we return
the result by inserting all the elements equal to first key, then the
second one until the last one.  And we obtain a stable sort.

Thus we add $\Theta(n)$ more space and running time to the algorithm.

\newpar{8.3-3} To prove that \textsc{radix-sort} is correct, let's
consider the following loop invariant:

\begin{quote}
  If we truncate the entries to the first $i-1$ digits, the array is
  sorted.
\end{quote}

\begin{itemize}
\item \textbf{Initialization: }  If $i=1$, truncating to the first $0$
  digits yields no number so we can say whatever properties we want on
  those numbers.

  \item \textbf{Maintenance: }  Suppose the property is true at the
    beginning of the \textbf{for} loop.

    We have the property that the elements of the array truncated to
    the first $i-1$ digits are in sorted order.  Now we sort the array
    $A$ on the $i^{th}$ digit using a stable sort.

    Consider two elements in the array.  If their respective $i^{th}$
    digits are different, then they are in the correct order.  If
    they are equal, since our sorting algorithm is a stable sort, they
    keep the same order as before the sorting.  Thus the elements of
    the array truncated to first $i$ digits is in sorted order.

    \item \textbf{Termination: } At the end of the \textbf{for} loop,
      we have $i = d+1$.  Thus the array is sorted.
\end{itemize}

\newpar{8.3-4}  Since the number of bits in each integer is
$\lg(n^2-1) \simeq 2 \lg n$, using \emph{Lemma 8.4} with $r = \lfloor
\lg n\rfloor$ we have a $\Theta(n)$ running time.

\newpar{8.3-5} Suppose we sort according to the first digit, it will
cost us $\Theta(n + k)$.  Note $n_i$ the number of numbers whose first
digit is equal to $i$.  Since we have to sort each of these piles of
numbers according to their second digits, it will cost us

\[ \sum_{i=1}^k \Theta(n_i + k) = \Theta(n + k^2).\]

The same reasoning holds true when sorting according to the third
digit which will cost us $\Theta(n + k^3)$ since there is $k^3$ piles
of numbers.   So the total cost will be

\[ \sum_{i=1}^d \Theta(n+k^i) = \Theta\left(d\,n +
k\frac{k^d-1}{k-1}\right) = \Theta(d\,n + k^d).\]

The number of piles of card the operator needs to keep track of in the
worst case is: $k\,d$.  To prove this claim, by symmetry we suppose
that sorting the numbers in the first pile in each turn use the worst
case memory.  And since we can only add $k$ piles of numbers of
sorting, we have the desired result.

\end{document}
