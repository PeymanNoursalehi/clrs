\documentclass[a4paper,12pt]{article}
\usepackage{algorithmic}
\newcommand{\newpar}[1]
{\bigskip \noindent \textbf{Exercises #1} \newline}
\newcommand{\newprob}[1]
{\bigskip \noindent \textbf{Problem #1} \newline}
\newcommand{\subpar}[1]{\medskip \noindent #1.}
\newcommand{\la}{\leftarrow}
\newcommand{\ra}{\rightarrow}
\newcommand{\prob}[1]{\mathrm{Pr}\left\{ #1 \right\}}

\begin{document}
\newpar{7.3-1}
Because we couldn't trick the algorithm to run on the worst-case
performance.

\newpar{7.3-2}
In the worst case the partition ratio would be $0$ to $1-1/n$ at each
step.  So \textsc{RANDOM} is called $\Theta(n)$ times.  In the best
case scenario,  the number of calls is $\Theta(\lg n)$ because the
ratio would be $1/2$ to $1/2$.
\end{document}
