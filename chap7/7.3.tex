\documentclass[a4paper,12pt]{article}
\usepackage{algorithmic}
\newcommand{\newpar}[1]
{\bigskip \noindent \textbf{Exercises #1} \newline}
\newcommand{\newprob}[1]
{\bigskip \noindent \textbf{Problem #1} \newline}
\newcommand{\subpar}[1]{\medskip \noindent #1.}
\newcommand{\la}{\leftarrow}
\newcommand{\ra}{\rightarrow}
\newcommand{\prob}[1]{\mathrm{Pr}\left\{ #1 \right\}}

\begin{document}
\newpar{7.3-1}
Because we couldn't trick the algorithm to run on the worst-case
performance.

\newpar{7.3-2} Let's note $T(n)$ the number of times \textsc{random}
is called when running the randomized \textsc{quicksort} algorithm on
an input array of size $n$.

Let's consider first the worst running time of
\textsc{randomized-quicksort} where at each step the partition ratio
would be $0$ to $1-1/n$. \textsc{random} is called $n-1$ times.  We
then deduce that $T(n) = \Omega(n)$.

Now let's show by induction that if $n \ge 1$, then $T(n) \le n-1$.
If $n = 1$, we don't call \textsc{random} thus $T(n) = 0$.  Suppose
that $n > 1$ and we have the property for all $k < n$.  Note $m$ the size
of the smaller partition we get after partitioning the array.  We have
$m < n$ and the size of the other partition is $n-m-1$.  Thus, by
induction
\[ T(n) = T(m) + T(n-m-1) + 1 = m-1 + n-m-2 + 1 \le n-1.\]

Hence $T(n) = O(n)$.  So finally $T(n) = \Theta(n)$.
\end{document}
