\documentclass[a4paper,12pt]{article}
\usepackage{algorithmic}
\newcommand{\newpar}[1]
{\bigskip \noindent \textbf{Exercises #1} \newline}
\newcommand{\newprob}[1]
{\bigskip \noindent \textbf{Problem #1} \newline}
\newcommand{\subpar}[1]{\medskip \noindent #1.}
\newcommand{\la}{\leftarrow}
\newcommand{\ra}{\rightarrow}
\newcommand{\prob}[1]{\mathrm{Pr}\left\{ #1 \right\}}

\begin{document}
\newpar{7.1-2}
When all the elements in the array $A[p..r]$ have the same value,  $r$
is returned by \textsc{PARTITION}.

\medskip \noindent
\textsc{PARTITION}($A$, $p$, $r$)
\begin{algorithmic}
  \STATE $q \la \lfloor \frac{p+r}{2}\rfloor$
  \STATE $x \la A[q]$
  \STATE $i \la p-1$
  \FOR{$j\la p$ to $r-1$}
  \IF{$j\le q$ \textbf{and}$A[j] \le x$ \textbf{or} $j > q$
    \textbf{and} $A[j]<x$}
  \STATE $i \la i+1$
  \STATE $\mathrm{exchange}\ A[i] \leftrightarrow A[j]$
  \ENDIF
  \ENDFOR
  \STATE $\mathrm{exchange}\ A[i+1] \leftrightarrow A[r]$
  \RETURN{$i+1$}
\end{algorithmic}

\newpar{7.1-3}
The \textbf{for} loop is executed $n$ times and at each iteration is a
$\Theta(1)$ operation.  Thus \textsc{PARTITION} is $\Theta(n)$.

\newpar{7.1-4}
Just change the test in line $4$ of \textsc{PARTITION} to $A[j] \ge x$.
\end{document}
