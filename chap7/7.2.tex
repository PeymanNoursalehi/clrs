\documentclass[a4paper,12pt]{article}
\usepackage{algorithmic}
\newcommand{\newpar}[1]
{\bigskip \noindent \textbf{Exercises #1} \newline}
\newcommand{\newprob}[1]
{\bigskip \noindent \textbf{Problem #1} \newline}
\newcommand{\subpar}[1]{\medskip \noindent #1.}
\newcommand{\la}{\leftarrow}
\newcommand{\ra}{\rightarrow}
\newcommand{\prob}[1]{\mathrm{Pr}\left\{ #1 \right\}}

\begin{document}
Let $c_1$ and $c_2$ be two positive constants such that
\[ T(n-1) + c_1\,n \le T(n) \le T(n-1) + c_2\,n\]
for $n > 0$. Let's show that
\[ K_1\,n^2\le T(n) \le K_2\,n^2\]
for appropriate choices of the positive constants $K_1$ and $K_2$.

Suppose we have the inequalities for $n-1$.  We have
\begin{eqnarray*}
  T(n) &\ge& T(n-1) + c_1\,n \\
  &\ge& K_1(n-1)^2 + c_1\,n \\
  &=& K_1n^2 + (c_1 - 2K_1)n + K_1{}^2 \\
  &\ge& K_1n^2
\end{eqnarray*}
if we have $K_1 \le \frac{c_1}{2}$.  And also
\begin{eqnarray*}
  T(n) &\le& T(n-1) + c_2\,n \\
  &\le& K_2(n-1)^2 + c_2\,n \\
  &=& K_2n^2 - \left((2K_2 - c_2) n - K_2\right)\\
  &\le& K_2n^2 - ((2K_2 - c_2) - K_2), \mbox{ if $K_2 \ge c_2$} \\
  &\le& K_2n^2
\end{eqnarray*}
Moreover,  if we choose $K_1$ small enough  and $K_2$ large enough
such that
\[ K_1 \le T(1) \le K_2,\]
we have the inequalities for $n > 0$.

\newpar{7.2-2}
In this case, the ratio of the partition is always $n-1$ to $0$.
Thus the running time verify the recurrence
\[ T(n) = T(n-1) + \Theta(n).\]

\newpar{7.2-3}
The ratio of the partition is always $0$ to $n-1$, so the same
reasoning applies.

\newpar{7.2-4}
If we note $i$ the number of inversions in an array of size $n$,  then
the running time of \textsc{INSERTION-SORT} is $\Theta(n+i)$.  Thus if
the array is almost sorted,  \textsc{INSERTION-SORT} would run in
linear time.

\newpar{7.2-5}
Given that $0 < \alpha \le \frac{1}{2}$ we have $0 < \alpha \le
1-\alpha.$  So the leaf having the minimum depth is leftmost one.  If
we note $h$ its height we have
\[ \alpha^h n \le 1 < \alpha^{h-1} n.\]
So approximatively the depth is $ - \log_\alpha n$.  The same
reasoning apply to the rightmost leaf whose heigh is $ -
\log_{1-\alpha}n$.

\newpar{7.2-6}
If we note $p$ the index of the partition in an array of size $n$,
this partition is more balanced than $1-\alpha$ to $\alpha$ if and
only if $\alpha\le \frac{p}{n}\le 1-\alpha$, that is
\[ \alpha\,n \le p\le (1-\alpha)\,n.\]
The index of the partition is $p$ if the pivot is the $p^{th}$ number
in increasing order.  Given that the input array is random,  the
probability to have a partition is
\[ \frac{(n-1)!}{n!} = \frac{1}{n}.\]
So the probability that the partition is more balanced than $1-\alpha$
to $\alpha$ is approximatively
\[\frac{\alpha\,n - (1-\alpha)n}{n} = 1- 2\alpha.\]
\end{document}
