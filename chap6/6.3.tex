\documentclass[a4paper,12pt]{article}
\usepackage{algorithmic}
\newcommand{\newpar}[1]
{\bigskip \noindent \textbf{Exercises #1} \newline}
\newcommand{\newprob}[1]
{\bigskip \noindent \textbf{Problem #1} \newline}
\newcommand{\subpar}[1]{\medskip \noindent #1.}
\newcommand{\la}{\leftarrow}
\newcommand{\ra}{\rightarrow}
\newcommand{\exchange}[2]{\mathrm{exchange}\ #1 \leftrightarrow #2}

\begin{document}

\newpar{6.3-2}
Because we want to ensure that each node $i+1, i+2, \ldots, n$ is the
root of a max-heap.  If we increase $i$ from $1$ to $\lfloor
\mathrm{length}[A]/2\rfloor$, this property is in general false
because we don't know if the left and right subtrees are max-heaps.

\newpar{6.3-3}
First, let's show the following lemma.

\begin{quote}
  If $a$ is a real and $n$ is a positive integer, then we have
  \[ \left\lfloor \frac{\left\lfloor a \right\rfloor}{n}
  \right\rfloor =
  \left\lfloor \frac{a}{n}\right\rfloor.\]
\end{quote}

First we have

\[\left\lfloor \frac{\left\lfloor a \right\rfloor}{n}
\right\rfloor \le
\frac{\left\lfloor a \right\rfloor}{n} \le \frac{a}{n}\]

Thus taking the floor we have the inequality

\[ \left\lfloor \frac{\left\lfloor a \right\rfloor}{n}
\right\rfloor \le
\left\lfloor \frac{a}{n}\right\rfloor.\,\mbox{(*)}\]

On the other hand we have

\[  n \left\lfloor\frac{a}{n}\right\rfloor \le a <
\left\lfloor a \right\rfloor + 1.\]

Since the leftmost and rightmost members of the inequalities are
integers, we then deduce

\begin{eqnarray*}
n \left\lfloor \frac{a}{n} \right\rfloor &\le& \lfloor a \rfloor \\
\left\lfloor \frac{a}{n} \right\rfloor &\le& \frac{\lfloor a\rfloor}{n} \\
\left\lfloor \frac{a}{n}\right\rfloor &\le& \left\lfloor
\frac{\lfloor a \rfloor}{n} \right\rfloor
\end{eqnarray*}

Thus with (*), we have the lemma.
Now consider an $n$-element heap.  Let's show by induction that
the nodes of height $h$ in the heap are:

\[ \left\lfloor
\frac{\left\lfloor\frac{n+2}{2}\right\rfloor}{2^h}\right\rfloor,
\left\lfloor
\frac{\left\lfloor\frac{n+2}{2}\right\rfloor}{2^h}\right\rfloor + 1,
\ldots, \left\lfloor \frac{n}{2^h} \right\rfloor.\]

If $h = 0$ then from \textbf{Exercises 6.1-7}, the leaves of the heap
are the nodes

\[ \left\lfloor \frac{n}{2} \right\rfloor + 1, \ldots, n.\]

Suppose we have the property for $h < \lfloor \lg n \rfloor$.  Then
the nodes of height $h+1$ are the parents of the nodes of height $h$.
By induction, we deduce that they're the nodes

\[\left\lfloor \frac{\left\lfloor
\frac{\left\lfloor\frac{n+2}{2}\right\rfloor}{2^h}\right\rfloor
}{2}\right\rfloor,
\ldots, \left\lfloor \frac{\left\lfloor \frac{n}{2^h}
  \right\rfloor}{2} \right\rfloor.\]

And with the lemma, we deduce the property for $h+1$.

\medskip
Keeping in mind that $-\lfloor x\rfloor = \lceil -x \rceil$, the
number of nodes of height $h$ is
\begin{eqnarray*}
  \left\lfloor \frac{n}{2^h} \right\rfloor -\left\lfloor
  \frac{\left\lfloor\frac{n+2}{2}\right\rfloor}{2^h}\right\rfloor + 1
  &=&
  \left\lfloor \frac{n}{2^h} \right\rfloor + \left\lceil
  -\frac{\left\lfloor\frac{n+2}{2}\right\rfloor}{2^h}\right\rceil + 1 \\
  &=& \left\lceil
  \left\lfloor \frac{n}{2^h} \right\rfloor - \frac{n}{2^h} +
  \frac{n + \left\lceil -\frac{n+2}{2}\right\rceil}{2^h}
  \right\rceil + 1 \\
  &=& \left\lceil
  \left\lfloor \frac{n}{2^h} \right\rfloor - \frac{n}{2^h} +
  \frac{\left\lceil \frac{n}{2}\right\rceil - 1}{2^h}
  \right\rceil + 1 \\
  &\le& \left\lfloor \frac{\left\lfloor \frac{n}{2}\right\rfloor -
    1}{2^h}\right\rfloor + 1 \\
  &\le& \left\lfloor \frac{\left\lfloor
    \frac{n}{2}\right\rfloor}{2^h}\right\rfloor + 1 \\
  &=& \left\lfloor \frac{n}{2^{h+1}}\right\rfloor + 1
\end{eqnarray*}
\end{document}
