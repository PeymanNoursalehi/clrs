\documentclass[a4paper,12pt]{article}
\usepackage{algorithmic}
\newcommand{\newpar}[1]
{\bigskip \noindent \textbf{Exercises #1} \newline}
\newcommand{\newprob}[1]
{\bigskip \noindent \textbf{Problem #1} \newline}
\newcommand{\subpar}[1]{\medskip \noindent #1.}
\newcommand{\la}{\leftarrow}
\newcommand{\ra}{\rightarrow}

\begin{document}
\newpar{5.1-1} We know a total order on the ranks of the candidates if
and only if two random candidates could always be compared to each
other.  Suppose we have more than two candidates and let $c_1$ and
$c_2$ be two of them.  Since the order of the candidates is random
there's a chance $c_1$ will be the first candidate and $c_2$ the
second one so in this case we could determine which one is better.

\newpar{5.1-2}
\textmd{RANDOM($a$, $b$)}
\begin{algorithmic}
  \STATE $max \la b-a$
  \STATE $res \la 0$
  \STATE $n \la 1$
  \WHILE{$n \le max$}
  \STATE $res \la res + n * \textmd{RANDOM(0, 1)}$
  \STATE $n \la 2 * n$
  \ENDWHILE
  \RETURN $a+res$
\end{algorithmic}
The expected running time of the procedure is
$\lfloor\lg(b-a)\rfloor$.

\newpar{5.1-3}
\textmd{RANDOM()}
\begin{algorithmic}
  \STATE $r_1 \la r_2 \la 0$
  \WHILE{ $r_1 = r_2$}
  \STATE $r_1 \la \textmd{BIASED-RANDOM()}$
  \STATE $r_2 \la \textmd{BIASED-RANDOM()}$
  \ENDWHILE
  \RETURN $r_1$
\end{algorithmic}
We have a geometric distribution with a probability $2p(1-p)$ of success.
So the expectation of the number of trials before we obtain a success
is $\frac{1}{2p(1-p)}$.
\end{document}
