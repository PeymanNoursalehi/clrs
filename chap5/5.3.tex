\documentclass[a4paper,12pt]{article}
\usepackage{algorithmic}
\newcommand{\newpar}[1]
{\bigskip \noindent \textbf{Exercises #1} \newline}
\newcommand{\newprob}[1]
{\bigskip \noindent \textbf{Problem #1} \newline}
\newcommand{\subpar}[1]{\medskip \noindent #1.}
\newcommand{\la}{\leftarrow}
\newcommand{\ra}{\rightarrow}
\newcommand{\prob}[1]{\mathrm{Pr}\left\{ #1 \right\}}

\begin{document}
\newpar{5.3-1}
\textsc{RANDOMIZE-IN-PLACE}($A$)
\begin{algorithmic}[1]
  \STATE $n \la length[A]$
  \STATE $\mathrm{swap}\ A[1] \leftrightarrow A[\textsc{RANDOM}(1,
    n)]$
  \FOR{$i\la 2$ to $n$}
  \STATE $\mathrm{swap}\ A[i] \leftrightarrow A[\textsc{RANDOM}(i,
    n)]$
  \ENDFOR
\end{algorithmic}
The invariant of the loop remains the same.  If $i=2$, the invariant
says that for each possible $1$-permutation, the subarray $A[1..1]$
contains this $1$-permutation with probability $1/n$.  This is true
because of line $2$.  And the rest of the proof is the same.

\newpar{5.3-2}
Suppose $n > 2$.  And consider the permutation where the first and
second elements of the array are swapped and everything else remains
in place.  Professor Kelp code couldn't produce this permutation
because each element couldn't stay at the same place.

\newpar{5.3-3}
Let's show that the probability to obtain a given permutation is
$1/n^n$ and not $1/n!$.  Consider the following invariant:
\begin{quote}
  Prior the $i^{th}$ iteration of the \textbf{for} loop, for each
  possible $(i-1)$-permutation, the subarray $A[1 .. i-1]$ contains
  this $(i-1)$-permutation with probability $1/n^{i-1}$.
\end{quote}
\begin{description}
\item[Initialization]
  For $i=1$ the subarray $A[1 .. 0]$ is an empty
  subarray and a $0$-permutation has no elements.  Thus $A[1 .. 0]$
  contains any $0$-permutation with probability $1$

\item[Maintenance]
  Consider an $i$-permutation $<x_1,\ldots,x_i>$.
  By hypothesis, the probability that the subarray $A[1.. i-1]$
  contains the $i-1$-permutation $<x_1,\ldots,x_{i-1}>$ is
  $1/n^{i-1}$.  So after we execute the body of the loop, the
  probability that the subarray $A[1 .. i]$ contains the
  $i$-permutation is 
  \[ \frac{1}{n^{i-1}} \times \frac{1}{n} = \frac{1}{n^i}.\]
  Incrementing $i$ then maintains the loop invariant.

\item[Termitantion]
  We have $i = n+1$, so the probability that the array $A[1..n]$
  contains any $n$-permutation is $1/n^n$.
\end{description}
Thus \textsc{PERMUTE-WILL-ALL} doesn't produce a uniform random
permutation.

\newpar{5.3-4}
Let $1\le i\le n$.  The probability of $A[i]$ winding up in $A[j]$ is
the probability that $offset$ is equal to $j - i \bmod n$ for every
$1\le j\le n$.  So it's equal to $1/n$.

An inversion couldn't be produced by \textsc{PERMUTE-BY-CYCLIC} so
this code doesn't produce an uniformly random permutation.

\newpar{5.3-5}
We have $(1-x)^n \ge 1 - nx$ for $0\le x\le 1$.  The
probability that all elements are unique is
\begin{eqnarray*}
  \frac{n^3}{n^3}\,\frac{n^3-1}{n^3}\ldots \frac{n^3-(n-1)}{n^3} &=&
  \prod_{i=0}^{n-1} \left(1 - \frac{i}{n^3}\right) \\
  &\ge& \prod_{i=0}^{n-1}\left(1 - \frac{1}{n^2}\right) \\
  &=& \left(1 - \frac{1}{n^2}\right)^n \\
  &\ge& 1 - n \frac{1}{n^2} \\
  &=& 1 - \frac{1}{n}
\end{eqnarray*}

\newpage
\newpar{5.3-6}
We just keep generating keys until there's no duplicate.

\medskip
\noindent
\textsc{PERMUTE-BY-SORTING}($A$)
\begin{algorithmic}
  \STATE $n \la length[A]$
  \STATE $i \la 1$
  \WHILE{$i \le n$}
  \STATE $dup \la \textsc{TRUE}$
  \WHILE{$dup$}
  \STATE $P[i] \la \textsc{RANDOM}(1, n^3)$
  \STATE $j \la i-1$
  \STATE $nodup \la \textsc{TRUE}$
  \WHILE{$nodup$\ \textbf{and}\ $j > 0$}
  \IF{$P[j] = P[i]$}
  \STATE $nodup \la \textsc{FALSE}$
  \ELSE
  \STATE $j \la j-1$
  \ENDIF
  \ENDWHILE
  \IF{$nodup$}
  \STATE $dup \la \textsc{FALSE}$
  \ENDIF
  \ENDWHILE
  \STATE $i \la i+1$
  \ENDWHILE
  \STATE \COMMENT{sort $A$, using $P$ as sort keys}
\end{algorithmic}
\end{document}
