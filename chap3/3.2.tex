\documentclass[a4paper,12pt]{article}
\usepackage{algorithmic}
\newcommand{\newpar}[1]
{\bigskip \noindent \textbf{Exercises #1} \newline}
\newcommand{\newprob}[1]
{\bigskip \noindent \textbf{Problem #1} \newline}
\newcommand{\subpar}[1]
{\medskip \noindent #1.}
\newcommand{\la}{\leftarrow}
\newcommand{\ra}{\rightarrow}

\begin{document}
\newpar{3.2-1}
Suppose that $f(n)$ and $g(n)$ are monotonically increasing
functions.  For $m \ge n$, we have
\[ f(n) \le f(m) \mbox{ and } g(n) \le g(m).\]
Taking the sum of the two inequalities, we have
\[ f(n) + g(n) \le f(m) + g(m).\]
And given that $f$ is monotone
\[ f(g(n)) \le f(g(m)).\]
If in addition, $f(n)$ and $g(n)$ are non negative, we could multiply
the two inequalities to obtain
\[ f(n) g(n)  \le f(m) g(m).\]

\newpar{3.2-2}
Let $a$, $b$, $c$ be real positive numbers, we have
\begin{eqnarray*}
\log_b(1/a) &=& (\log_ba + \log_b(1/a)) - \log_b a \\
&=& \log_b\left(a \times \frac{1}{a}\right) - \log_b a \\
&=& \log_b 1 - \log_b a \\
&=& -\log_b a
\end{eqnarray*}
And 
\begin{eqnarray*}
\log_ba &=& \frac{\ln a}{\ln b} \\
&=& \left(\frac{\ln b}{\ln a}\right)^{-1} \\
&=& \frac{1}{\log_a b}
\end{eqnarray*}
Lastly,
\begin{eqnarray*}
a^{\log_b c} &=& e^{\frac{\ln b}{\ln a} \ln c} \\
&=& e^{\ln b \ln_a c} \\
&=& b^{\ln_a c}
\end{eqnarray*}

\newpar{3.2-3}
Using Stirling's approximation, we have
\[ \frac{n!}{n^n} = \sqrt{2\pi n}\frac{1}{e^n}
\left(1 + \Theta\left(\frac{1}{n}\right)\right).\]
Thus,
\[ \lim_{n \to +\infty}\frac{n!}{n^n} = 0.\]
So $n! = o(n^n)$.  And we have,
\[ \lim_{n \to +\infty}\frac{n!}{2^n} =
\lim_{n \to +\infty}\sqrt{2\pi n}\left(\frac{n}{2e}\right)^n
\left(1 + \Theta\left(\frac{1}{n}\right)\right) = +\infty
\]
so $n! = \omega(2^n)$.  And finally,
\begin{eqnarray*}
\lg(n!) &=& n \lg n - \lg e + \frac{1}{2} (\log(2\pi) + n)
+ \lg\left(1 + \Theta\left(\frac{1}{n}\right)\right) \\
&=& \Theta(n \lg n)
\end{eqnarray*}

\newpar{3.2-4}
Using Stirling's formula, we have
 \[\lceil \lg n \rceil ! = \sqrt{2\pi \lceil \lg n \rceil}
\left(\frac{\lceil \lg n\rceil}{e}\right)^{\lceil \lg n\rceil}
\left(1 + \Theta\left(\frac{1}{\lg n}\right)\right).\]
But we have,
\begin{eqnarray*}
\left(\frac{\lceil \lg n\rceil}{e}\right)^{\lceil \lg n \rceil} &\ge&
\left(\frac{\lg n}{e}\right)^{\lg n} \\ &=&
2^{\lg n (\lg \lg n - \lg e)} \\ &=&
n^{\lg \lg n - \lg e}
\end{eqnarray*}
So $\lceil \lg n \rceil !$ couldn't be polynomially bounded.

Taking the same reasonning, we have
\begin{eqnarray*}
\left(\frac{\lceil \lg \lg n\rceil}{e}\right)^{\lceil \lg \lg
  n\rceil}&\le&
\left(\frac{\lg \lg n + 1}{e}\right)^{\lg \lg n + 1} \\ &=&
(\lg \lg n + 1)^{\lg \lg n + 1} \frac{1}{e (\lg n)^{\lg e}} \\ &=&
(\lg \lg n + 1)^{\lg \lg n} 
\frac{\lg \lg n + 1}{e (\lg n)^{\lg e}} \\ &=&
(\lg \lg n)^{\lg \lg n}\left(1 + \frac{1}{\lg \lg n}\right)^{\lg \lg
  n} \frac{\lg \lg n + 1}{e (\lg n)^{\lg e}} \\&=&
(\lg \lg n)^{\lg \lg n}\ O(1) \\ &=&
2^{\lg \lg n \times \lg \lg \lg n} O(1)
\end{eqnarray*}
We have $\lg \lg \lg n = O(\lg \lg n) = O(\lg n/\lg \lg n)$.  So
there exists $c > 0$ and $n_0$ positive integer such that for $n >
n_0$, we have
\[ \lg \lg \lg n \le c \frac{\lg n}{\lg \lg n}.\]
We then deduce that for $n > n_0$,
\[\left(\frac{\lceil \lg \lg n\rceil}{e}\right)^{\lceil \lg \lg
  n\rceil}\le O(n^c).\]
Plus, we have
\[ \sqrt{2\pi \lceil \lg \lg n\rceil} = O(n),\]
and
\[\left(1 + \Theta\left(\frac{1}{\lceil \lg \lg n\rceil}\right)
\right) = O(1).\]
So from Stirling's formula, we deduce
\begin{eqnarray*}
\lceil \lg \lg n\rceil ! &\le& O(n)\ n^c\ O(1) \\
&=& O(n^{c+1})
\end{eqnarray*}

\newpar{3.2-5}
For large values of $n$, we have
\begin{eqnarray*}
\lg^* \lg n &=& \min\{i \ge 0: \lg^{(i)}\lg n \le 1\} \\
&=& \min\{i \ge 0: \lg^{(i+1)} n \le 1\} \\
&=& \min\{i > 0: \lg^{(i)} n \le 1\} \\
&=& \lg^* n
\end{eqnarray*}
So $\lg^*(\lg n)$ is asymptotically larger than than $\lg(\lg^*n)$.

\newpar{3.2-6}
We have,
\[ F_0 = 0 = \frac{\phi^0 - {\hat{\phi}}^0}{\sqrt{5}},\]
and
\[ F_1 = 1 = \frac{\phi - \hat{\phi}}{\sqrt{5}}.\]

\medskip  Let $i$ be a non-negative integer and suppose that
we have the equalities for $i$ and $i+1$.  We have,
\begin{eqnarray*}
F_{i+2} &=& F_{i+1} + F_i \\
&=& \frac{\phi^{i+1} - {\hat{\phi}}^{i+1}}{\sqrt{5}} +
\frac{\phi^i - {\hat{\phi}}^i}{\sqrt{5}}, \mbox{ by induction} \\
&=& \frac{\phi^i(\phi+1) - {\hat{\phi}}^i(\hat{\phi} + 1)}
{\sqrt{5}} \\
&=& \frac{\phi^i \phi^2 - {\hat{\phi}}^i {\hat{\phi}}^2}{\sqrt{5}} \\
&=& \frac{\phi^{i+2} - {\hat{\phi}}^{i+2}}{\sqrt{5}} \\
\end{eqnarray*}

\newpar{3.2-7}
We have,
\[ F_2 = 1 \ge \phi^0\ \mbox{and}\ F_3 = 2 \ge \phi.\]
Suppose that for $i \ge 0$, we have
\[ F_{i+2} \ge \phi^i\ \mbox{and}\ F_{i+3} \ge \phi^{i+1}.\]
Thus, we have
\begin{eqnarray*}
F_{i+4} &=& F_{i+3} + F_{i+2} \\
&\ge& \phi^{i+1} + \phi^i,\ \mbox{by induction} \\
&=& \phi^i(\phi+1) \\
&=& \phi^i\phi^2\\
&=& \phi^{i+2}
\end{eqnarray*}

\newprob{3-1}
a. Suppose $k \ge d$, then for $n \ge 1$ we have
\begin{eqnarray*}
p(n) &\le& n^d\left(\sum_{i=0}^da_i\right)\ \mbox{(*)}\\
&\le& n^k \left(\sum_{i=0}^da_i\right)
\end{eqnarray*}
Thus, $p(n) = O(n^k)$.

\medskip \noindent b. Suppose $k \le d$, then for $n \ge 1$ we have
for $0 \le i \le d-1$
\[ a_i n^i \ge - |a_i| n^i \ge -|a_i| n^{d-1}.\]
We then deduce that,
\begin{eqnarray*}
p(n) &\ge& n^d\left(a_d - \sum_{i=0}^{d-1} \frac{|a_i|}{n}\right)
\end{eqnarray*}
For $n \ge 2 \sum_{i=0}^{d-1}|a_i|/a_d$, we have
\[a_d - \sum_{i=0}^{d-1} \frac{|a_i|}{n} \ge \frac{a_d}{2}.\]
We then have
\begin{eqnarray*}
p(n) &\ge& \frac{a_d}{2} n^d\ \mbox{(**)}\\ 
&\ge& \frac{a_d}{2} n^k.
\end{eqnarray*}
Finally, we conclude that $p(n) = \Omega(n^k)$.

\medskip \noindent
c. From a. and b., we deduce that if $k=d$ $p(n) = \Theta(n^k)$.

\medskip \noindent
d. We deduce easily from (*) that if $k > d$, then $p(n) = o(n^k)$.

\medskip \noindent
e. From (**), we deduce that if $k < d$, then $p(n) = \omega(n^k)$.

\newprob{3-2}
\begin{tabular}{|c|c|c|c|c|c|c|c|}
\hline
&$\mathbf{A}$&$\mathbf{B}$&$\mathbf{O}$&$\mathbf{o}$&
$\mathbf{\Omega}$&$\mathbf{\omega}$&$\mathbf{\Theta}$\\ 
\hline
\textbf{a.}&$\lg^kn$&$n$&yes&yes&no&no&no\\ 
\hline
\textbf{b.}&$n^k$&$c^n$&yes&yes&no&no&no\\
\hline
\textbf{c.}&$\sqrt{n}$&$n^{\sin n}$&no&no&no&no&no\\
\hline
\textbf{d.}&$2^n$&$2^{n/2}$&no&no&yes&yes&no\\
\hline
\textbf{e.}&$n^{\lg c}$&$c^{\lg n}$&yes&no&yes&no&yes\\
\hline
\textbf{f.}&$\lg(n!)$&$\lg(n^n)$&yes&no&yes&no&yes\\
\hline
\end{tabular}

\medskip
Let's explain only for \textbf{c.}, the others are straightforward.

Let $l$ be a positive integer.  Given that $\pi/2 > 1$, there exists
two integers $n$ and $m$ such that,
\[ \frac{\pi}{4} + 2l\pi < n < \frac{3\pi}{4} + 2l\pi,\]
and
\[ \frac{5\pi}{4} + 2l\pi < m < \frac{7\pi}{4} + 2l\pi.\]
We then deduce that,
\[ \frac{1}{2} < \sin n < 1,\]
and
\[ -1 < \sin m < -\frac{1}{2}.\]
$n$ and $m$ could be arbitraly large so we can't compare the order of
growth of $n^k$ with $n^{\sin n}$.

\newprob{3-3}
a. We have,
\begin{eqnarray*}
2^{2^{n+1}} &=& \Omega(2^{2^n}) \\
2^{2^n} \ &=& \Omega((n+1)!)\\
(n+1)! &=& \Omega(n!) \\
n! &=& \Omega(e^n ) \\
e^n  &=& \Omega(n \times 2^n) \\
n \times 2^n &=& \Omega\left(2^n\right)\\
2^n &=& \Omega\left(\left(\frac{3}{2}\right)^n\right) \\
\left(\frac{3}{2}\right)^n &=& \Omega\left(n^{\lg \lg n}\right) \\
n^{\lg \lg n} &=& \Omega\left((\lg n)!\right) \\
(\lg n)! &=& \Omega\left((\lg n)^{\lg n}\right) \\
(\lg n)^{\lg n} &=& \Omega\left(n^3\right) \\
n^3 &=& \Omega\left(4^{\lg n}\right) \\
4^{\lg n} &=& \Omega\left(n^2\right) \\
n^2 &=& \Omega\left(\lg(n!)\right) \\
\lg(n!) &=& \Omega\left(n \lg n\right) \\
n \lg n &=& \Omega\left(2^{\lg n}\right) \\
2^{\lg n} &=& \Omega\left(n\right) \\
n &=& \Omega\left(\left(\sqrt{2}\right)^{\lg n}\right) \\
\left(\sqrt{2}\right)^{\lg n} &=& \Omega\left(2^{\sqrt{2\lg n}}\right) \\
2^{\sqrt{2\lg n}} &=& \Omega\left(\lg^2 n\right) \\
\lg^2 n &=& \Omega\left(\ln n\right) \\
\ln n &=& \Omega\left(\sqrt{\lg n}\right) \\
\sqrt{\lg n} &=& \Omega\left(\ln \ln n\right) \\
\ln \ln n &=& \Omega\left(2^{\lg^* n}\right) \\
2^{\lg^* n} &=& \Omega\left(\lg^* n\right) \\
\lg^* n &=& \Omega\left(\lg^*(\lg n)\right) \\
\lg^*(\lg n) &=& \Omega\left(\lg(\lg^* n)\right) \\
\lg(\lg^* n) &=& \Omega\left(n^{1/\lg n}\right) \\
n^{1/\lg n} &=& \Omega(1)
\end{eqnarray*}

\medskip \noindent
b. For $n$ positive integer, let $q$ and $r$ be the quotient and
remainder of the division of $n$ by $30$.
Let's consider the functions $\phi$ and $\psi_i$ defined by
\[ \phi(n) = \left\{
\begin{array}{cl}
0 & \mbox{if $q$ is even}, \\
n & \mbox{otherwise}
\end{array}\right.\]
And for $1 \le i \le 30$,
\[ \psi_i(n) = \left\{
\begin{array}{cl}
1 & \mbox{if $r = i-1$} \\
0 & \mbox{otherwise}
\end{array} \right. \]
And finally, the function $f$ defined as
\[ f(n) = \phi(n) \sum_{i=1}^{30} \psi_i(n) g_i(n) .\]

For $1 \le i \le 30$, and $k$ integer we have
\[ \phi(60 k + i-1) = 0 \mbox{ and } 
\phi(30(2k+1)+i-1) = n.\]
And for $1 \le j \le 30$
\[ \psi_j(30(2k+1)+i-1) = \left\{
\begin{array}{cl}
1 & \mbox{if $i = j$}, \\
0 & \mbox{otherwise}
\end{array} \right. .\]
So we have,
\[ f(60k+i-1) = 0 = o(g_i(60k+i-1)), \]
and
\begin{eqnarray*}
f(30(2k+1)+i-1) &=& (30(2k+1)+i-1)g_i(30(2k+1)+i-1) \\
&=& \omega(g_i(30(2k+1)+i-1)).
\end{eqnarray*}
We then deduce that $f$ is neither $O(g_i(n))$ nor $\Omega(g_i(n))$
for all $i$ between $1$ and $30$.

\newprob{3-4}
\subpar{a} False.  We have $n = O(n^2)$ but $n^2 \not= O(n)$.

\subpar{b} False. We have $\min(n, n^2) = n$ and $n + n^2 =
\Theta(n^2) \not= \Theta(n)$.

\subpar{c} Suppose $f(n) = O(g(n))$.  There exists $C > 0$ such that
for sufficiently large $n$
\[ f(n) \le C g(n).\]
We then deduce
\begin{eqnarray*}
\lg f(n) &\le& \lg C + \lg g(n) \\
&=& (1+\lg C)\lg g(n) - ((\lg g(n) - 1) \lg C)
\end{eqnarray*}
But for sufficiently large $n$, $\lg g(n) \ge 1$ by hypothese.  Plus
we can choose $C$ to be greater than $2$. Thus,
\[ \lg f(n) \le (1+\lg C)\lg g(n).\]
So $\lg f(n) = O(\lg g(n))$.

\subpar{d} False.  We have $2n = O(n)$ but $2^{2n} = 4^n \not=
O(2^n)$.

\subpar{e} False.  $1/n \not= O(1/n^2)$.

\subpar{f} Suppose $f(n) = O(g(n))$.  There exists $C > 0$ such that
for sufficiently large $n$ we have,
\[ f(n) \le C g(n).\]
Thus,
\[ \frac{1}{C} f(n) \le g(n).\]
Then, we can deduce that $g(n) = \Omega(f(n))$.

\subpar{g}  False. We have, $2^{n/2} = o(2^n)$, thus $2^n \not=
  \Theta(2^{n/2})$.

\subpar{h}  Let $g$ be function such that $g(n) = o(f(n))$.  For
sufficiently large $n$, we have
\[ g(n) \le f(n).\]
Thus,
\[ f(n) + g(n) \le 2f(n).\]
We then deduce that $f(n) + g(n) = O(f(n))$.  Since $g$ is
asymptotically positive, we have easily $f(n) + g(n) = \Omega(f(n))$.

So we deduce that $f(n) + o(f(n)) = \Theta(f(n))$.

\newprob{3-5}
\subpar{a} Let $f$ and $g$ be two asymptotically nonnegative
functions.  And suppose $f(n) \not= O(g(n))$.

We then deduce that for all interger $N$ there exists $n > N$ such that
\[ f(n) \ge g(n).\]
In particular, there exists $n_0 > 0$ such that,
\[ f(n_0) \ge g(n_0).\]
Suppose that we have a sequel $n_0\le n_1\le \ldots\le n_k$ such that
\[ f(n_l) \ge g(n_l)\ \mbox{ for $0 \le l \le k$}.\]
Then, there exists $n_{k+1} > n_k$ such that,
\[ f(n_{k+1} \ge g(n_{k+1}).\]
So $f(n) \ge g(n) \ge 0$ for infinitely many intergers $n$.  We then
deduce that $f(n) =\ \stackrel{\infty}{\Omega}(g(n))$.

\medskip
On the other side, consider the function $f$ defined by,
\[ f(n) = \left\{
\begin{array}{cl}
0 & \mbox{if $n$ is even} \\
1 & \mbox{otherwise}
\end{array} \right. \]
and $g(n) = f(n+1)$.  For all constants $C > 0$ we have,
\[ f(2n) < Cg(2n)\ \mbox{and}\ Cg(2n+1) < f(2n+1).\]
Thus, $f(n) \not= O(g(n))$ and $f(n) \not= \Omega(g(n))$.

\subpar{b}  Note $T$ the running time of an algorithm.  Suppose that
there exists $C > 0$ such that sufficiently large $n$ of the form 
$2^l(2m+1)$ we have,
\[ 0 \le Cg(n) \le T(n).\]
This is really informative because only the intergers wich are exact
power of $2$ are excluded.  But theses numbers become rarer as $n$
becomes larger.  There's only $\lfloor \lg N\rfloor + 1$ exact power
of $2$ less than $N$.

On the other hand, if we have the previous property only for integers
wich are exact power of $2$, it's useless.  Because we have no
information for the entries we would most likely encounter.

So basically, the $\stackrel{\infty}{\Omega}$ is useful only the
proportion of the integers wich don't verify the property is
negligible compared to those which do.

\subpar{c}  Given that $f(n)$ is asymptotically positive, for sufficiently
large $n$, we have $|f(n)| = f(n)$.  Thus \emph{Theorem 3.1} holds.

\subpar{d} Similarly we have,
\[ \tilde{\Omega}(g(n)) = \{f(n): \exists\ c, k, n_0\ \mbox{such
  that}\ 0 \le c\,g(n) \lg^k(n) \le f(n)\ \forall\ n \ge n_0\}.\]
and,
\begin{eqnarray*}
\tilde{\Theta}{g(n)} = \{f(n)&:& \exists\ c_1, c_2, k, n_0\ 
\mbox{such that}\\
&& 0 \le c_1\,g(n) \lg^k(n) \le f(n) \le
c_2\,g(n) \lg^k(n)\ \forall\ n \ge n_0\}.
\end{eqnarray*}
The same reasonning as in \emph{Exercises 3.1-5} apply to prove the
equivalent of \emph{Theorem 3.1}.

\newprob{3-6}
\begin{tabular}{|c|c|c|c|}
\hline
&$\mathbf{f(n)}$&$\mathbf{c}$&$\mathbf{f_c^*(n)}$\\
\hline
\textbf{a.}&$n-1$&$0$&$n$\\
\hline
\textbf{b.}&$\lg n$&$1$&$\lg^*n$ \\
\hline
\textbf{c.}&$n/2$&$1$&$\lceil \lg n\rceil$ \\
\hline
\textbf{d.}&$n/2$&$2$&$\lceil \lg n\rceil - 1$ \\
\hline
\textbf{e.}&$\sqrt{n}$&$2$&$\lceil \lg \lg n\rceil$ \\
\hline
\textbf{f.}&$\sqrt{n}$&$1$&$+\infty$ \\
\hline
\textbf{g.}&$n^{1/3}$&$1$&$\lceil \log_3 \lg n\rceil$ \\
\hline
\textbf{h.}&$n/\lg n$&$2$&$\Theta(\lg n / \lg \lg n)$ \\
\hline
\end{tabular}

\medskip
Only the last one is subtil.  Let's show by induction that for all integers
$1\le k < \lg n/\lg \lg n$, we have
\[ \frac{n}{\lg^k n} \le f^{(k)}(n).\]
We have this property for $k=1$, because
\[ f(n) = \frac{n}{\lg n}.\]
Suppose that we have this property for $k < \lg n/\lg \lg n$.  $f$
is monotonically increasing function, thus we have
\[ f\left(\frac{n}{\lg^k n}\right) \le f^{(k+1)}(n).\]
We have,

\begin{eqnarray*}
f\left(\frac{n}{\lg^k n}\right) &=&
\frac{n}{\lg^k n\ (\lg n - k \lg \lg n)} \\
&=&\frac{n}{\lg^{k+1}n\ (1 - k \frac{\lg \lg n}{\lg n})} \\
&\ge& \frac{n}{\lg^{k+1}n}\ \mbox{because $k < \lg n /\lg \lg n$}
\end{eqnarray*}
We then deduce that,
\[ \frac{n}{\lg^{k+1}n} \le f^{(k+1)}(n).\]
So far we have for $k < \lg n / \lg \lg n$,
\[ \frac{n}{\lg^kn} \le f^{(k)}(n).\]
But for $k < (\lg n - 1)/\lg \lg n$, we have
\[ 2 < \frac{n}{\lg^k n} \le f^{(k)}(n).\]
We then deduce that,
\[ f^*_2(n) \ge \frac{\lg n - 1}{\lg \lg n}.\]
Thus $f^*_2(n) = \Omega(\lg n/\lg\lg n)$.

\medskip
Now, let's show by induction for that $0 \le k < \lg n/\lg \lg n + 1$
\[ f^{(k)}(n) \le \frac{n}{\prod_{i=0}^{k-1}(\lg n - i \lg \lg n)}.\]
The case is trivial for $k=0$.  Suppose we have the property for $k$.
Given that $f$ is an increasing function, we have,
\begin{eqnarray*}
f^{(k+1)}(n) &\le& f\left(\frac{n}
{\prod_{i=0}^{k-1}(\lg n - i \lg \lg n)}\right) \\
&=& \frac{n}{\prod_{i=0}^{k-1}(\lg n - i \lg \lg n)} \times
\frac{1}{\lg n - \sum_{i=0}^{k-1}\lg (\lg n - i \lg \lg n)} \\
&=& \frac{n}{\prod_{i=0}^{k-1}(\lg n - i \lg \lg n)}\,
\frac{1}{\lg n - k \lg \lg n - 
\sum_{i=0}^{k-1}\lg \left(1 - i \frac{\lg \lg n}{\lg n}\right)}
\end{eqnarray*}
Given that $0 < k < \lg n / \lg \lg n + 1$, we have
\[ - \sum_{i=0}^{k-1}\lg(1 - i \lg n/\lg \lg n) > 0\]
because each term is positive.
We then deduce that,
\[ f^{(k+1)}(n) \le \frac{n}{\prod_{i=0}^{k}(\lg n - i \log \lg n)}.\]

So for $0 < k < \lg / \lg \lg n + 1$, we have
\begin{eqnarray*}
f^{(k)}(n) &\le&\frac{n}
{(\lg \lg n)^k \prod_{i=0}^{k-1}(\lg n/\lg \lg n - i)} \\
&\le& \frac{n}
{(\lg\lg n)^k \prod_{i=0}^{k-1}(\lfloor \lg n/\lg\lg n\rfloor - i)}
\end{eqnarray*}
Thus, for $k = \lfloor \lg n/\lg\lg n\rfloor$ we have
\[ f^{(k)}(n) \le \frac{n}{(\lg\lg n)^k\lfloor \lg n/\lg\lg
  n\rfloor!}.\]
But for $m \ge 1$, we have
\[ m! = \sqrt{2\pi m}\left(\frac{m}{e}\right)^m e^{\alpha_m}\]
where
\[ \frac{1}{12m+1} < \alpha_m < \frac{1}{12m}.\]
We then deduce that
\[ \frac{1}{m!} < \left(\frac{e}{m}\right)^m.\]
Thus,
\begin{eqnarray*}
f^{(k)}(n) &\le& \frac{n\,e^{\lfloor \lg n/\lg\lg n\rfloor}}{
(\lg\lg n\ \lfloor \lg n/\lg\lg n\rfloor)^{\lfloor \lg n/\lg\lg
    n\rfloor}} \\ &\le&
\frac{n\,e^{\lg n/\lg\lg n}}{(\lg\lg n\ \lfloor \lg n/\lg\lg n\rfloor)^{\lfloor \lg n/\lg\lg
    n\rfloor}}
\end{eqnarray*}
Given that $\lfloor \lg n/\lg\lg n\rfloor > \lg n/\lg\lg n - 1$,
we have
\begin{eqnarray*}
(\lg\lg n \lfloor \lg n/\lg\lg n\rfloor)^{\lfloor \lg n/\lg \lg
    n\rfloor}
&\ge& (\lg n - \lg\lg n)^{\lg n/\lg\lg n - 1} \\
&=& 2^{(\lg n / \lg\lg n - 1)\lg(\lg n - \lg \lg n)} \\
&=& 2^{(\lg n/\lg\lg n - 1)(\lg \lg n + \lg(1 - \lg \lg n/\lg n))} \\
&\ge& 2^{(\lg n - \lg\lg n)}\,2^{\lg n/\lg\lg n \lg(1-\lg\lg n/\lg n)} \\
&=& \frac{n}{\lg n}\,2^{\lg n/\lg\lg n \lg(1-\lg\lg n/\lg n)}
\end{eqnarray*}
We have,
\[ \lim_{n \to +\infty}\,2^{\lg n/\lg\lg n \lg(1-\lg\lg n/\lg n)}
= 2^{-1/\ln 2} \simeq 0.37.\]
So for sufficiently large $n$ we have
\[ (\lg\lg n \lfloor \lg n/\lg\lg n\rfloor)^{\lfloor \lg n/\lg \lg
    n\rfloor}
\ge \frac{n}{2\lg n}.\]
So finally for $k = \lfloor\lg n/\lg\lg n\rfloor$, we have for
sufficiently large $n$
\[ f^{(k)}(n) \le 2\lg n\,e^{\lg n/\lg\lg n} \le
e^{2\lg n/\lg\lg n}.\ \mbox{(*)}\]
For sufficiently large $n$, let's show by induction that for $\alpha
= o(\lg\lg n)$, there exists an integer $k$ such that
\[ k \le \frac{\lg n}{\lg\lg n} +
\sum_{i=1}^\alpha \lg n\left(\frac{4}{\lg\lg n}\right)^i
\ \mbox{(1)}\]
and
\[ f^{(k)}(n) \le \exp\left(\lg n\,
\left(\frac{4}{\lg\lg n}\right)^\alpha\right)\ \mbox{(2)}.\]

\medskip
For $\alpha = 1$ we deduce these inequalities from (*).  Suppose we
have these properties for $\alpha \ge 1$.

Note $N = \exp\left(\lg n\, \left(\frac{4}{\lg\lg
  n}\right)^\alpha\right)$.  From (*) we have
\[ f^{(\lfloor \lg N/\lg\lg N\rfloor)}(N) \le
e^{2\lg N/\lg\lg N}\ \mbox{(**)}.\]
We have,
\[\frac{2\lg N}{\lg\lg N} =
2\lg n\left(\frac{4}{\lg\lg n}\right)^{\alpha} \frac{\lg e}{\lg\lg n
  +\lg\lg e + \alpha(2 - \lg\lg\lg n)}.\] Given that $\lg e < 2$, if
$\alpha = o(\lg\lg n)$ we have for sufficiently large $n$
\[\frac{\lg e}{1 + 
\frac{\lg\lg e + \alpha(2- \lg\lg\lg n)}{\lg\lg n}} \le 2.\]
Thus, 
\[\frac{2\lg N}{\lg\lg N} \le 
\lg n \left(\frac{4}{\lg\lg n}\right)^{\alpha+1}\ \mbox{(***)}.\] Note
$k' = k + \lfloor \lg N/\lg\lg N\rfloor$.  From (1) and (***), we
deduce easily that
\[ k' \le \frac{\lg n}{\lg\lg n} +
\sum_{i=1}^{\alpha+1} \lg n\left(\frac{4}{\lg\lg n}\right)^i.\]
And from (2), we deduce given that $f$ is a monotically
increasing function

\begin{eqnarray*}
f^{(k')}(n) &=& f^{(\lfloor \lg N/\lg\lg N\rfloor)}
\left(f^{(k)}(n)\right) \\
&\le& f^{(\lfloor \lg \lg N/\lg\lg N\rfloor)}(N) \\
&\le& e^{2\lg N/\lg\lg N},\ \mbox{from (**)} \\
&\le&\exp\left(\lg n\,
\left(\frac{4}{\lg\lg n}\right)^{\alpha+1}\right)\ \mbox{from (***)}
\end{eqnarray*}

\medskip
So, we demonstrated that the properties (1) and (2) are verified
for sufficiently large $n$ and if $\alpha = o(\lg\lg n)$.  But for 
\[ \alpha \ge \frac{\lg\lg n - \lg \ln 2}{\lg\lg\lg n - 2}\]
we have
\[ \exp\left(\lg n \left(\frac{4}{\lg\lg n}\right)^\alpha\right)
\le 2.\]
So for 
$\alpha = \left\lceil \frac{\lg\lg n - \lg \ln 2}{\lg\lg\lg n - 2}
\right\rceil = o(\lg\lg n)$, there exists an integer $k$ such that
\[ k \le \frac{\lg n}{\lg\lg n} +
\sum_{i=1}^\alpha \lg n\left(\frac{4}{\lg\lg n}\right)^i\]
and
\[ f^{(k)}(n) \le 2.\]
We then deduce that,
\begin{eqnarray*}
f^*_2(n) &\le& k \le \frac{\lg n}{\lg\lg n} +
\sum_{i=1}^\alpha \lg n\left(\frac{4}{\lg\lg n}\right)^i \\
&\le&\frac{\lg n}{\lg\lg n} +
\sum_{i=1}^{+\infty} \lg n\left(\frac{4}{\lg\lg n}\right)^i \\
&=&\frac{\lg n}{\lg\lg n} +
\frac{4\lg n}{\lg\lg n(1 - 4/\lg\lg n)}
\end{eqnarray*}
We then deduce that $f^*_2(n) = O(\lg n/\lg\lg n)$.  So finally
\[ f^*_2(n) = \Theta\left(\frac{\lg n}{\lg\lg n}\right).\]
\end{document}
