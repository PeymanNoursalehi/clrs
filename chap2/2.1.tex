\documentclass[a4paper,12pt]{article}
\usepackage{algorithmic}
\newcommand{\newpar}[1]
{\bigskip \noindent \textbf{Exercises #1} \newline}
\newcommand{\newprob}[1]
{\bigskip \noindent \textbf{Problem #1} \newline}
\newcommand{\la}{\leftarrow}
\newcommand{\ra}{\rightarrow}
\begin{document}

\newpar{2.1-1} The current sorted array is in bold.
\begin{eqnarray*}
&\ra&\mathbf{31}\ 41\ 59\ 26\ 41\ 58\\&\ra&
\mathbf{31\ 41}\ 59\ 26\ 41\ 58\\&\ra&
\mathbf{31\ 41\ 59}\ 26\ 41\ 58\\&\ra&
\mathbf{26\ 31\ 41\ 59}\ 41\ 58\\&\ra&
\mathbf{26\ 31\ 41\ 41\ 59}\ 58\\&\ra&
\mathbf{26\ 31\ 41\ 41\ 58\ 59}
\end{eqnarray*}

\newpar{2.1-2}
INSERTION-SORT($A$)
\begin{algorithmic}
\FOR{$j \la 2$ to $length[A]$}
	\STATE $key \la A[j]$
	\STATE $i \la j-1$
	\WHILE{$i > 0$ and $A[i] < key$}
		\STATE $A[i+1] \la A[i]$
		\STATE $i \la i-1$
	\ENDWHILE
	\STATE $A[i+1] \la key$
\ENDFOR
\end{algorithmic}

\newpar{2.1-3}
LINEAR-SEARCH($A$, $v$)
\begin{algorithmic}
\STATE $r \la \mathbf{NIL}$
\STATE $i \la 1$
\WHILE {$r = \mathbf{NIL}$ and $i \le length[A]$}
	\IF {$A[i] = v$}
		\STATE $r \la i$
	\ENDIF
	\STATE $i \la i+1$
\ENDWHILE
\RETURN $r$
\end{algorithmic}

Let's use the following property as loop invariant:
\[ r = \mathbf{NIL} \mbox{ and } A[k] \not= v, 1\le k \le i-1.\]
\begin{itemize}
\item
\textbf{Initialization: }For $i=1$, the subarray $A[1\ \ i-1]$ is empty
so the property holds.
\item
\textbf{Maintenance: }Suppose we have the property from the previous
iteration of the loop and we have:
\[ r = \mathbf{NIL} \mbox{ and } i \le length[A].\]
Since $r = \mathbf{NIL}$, we have $A[i-1] \not= b$ otherwise we would
have $r = i-1$. Thus $A[k] \not= v$ for $1\le k\le i-1$.
\item
\textbf{Termination: }If we have $i = length[A]+1$ at the end of the
loop, we deduce the property
\[ A[k] \not= v \mbox{ for } 1\le k\le length[A].\]
So $v$ is not in the array $A$.  Otherwise, the loop termination is
caused by the fact that $r \not= \mathbf{NIL}$.  We deduce easily from
the body of the loop that $v = A[r]$.
\end{itemize}

\newpar{2.1-4}
\textbf{Input: } Two $n$-element arrays $A$ and $B$ representing binary
numbers.\\
\textbf{Ouput: } $(n+1)$-element array $C$ representing the sum of $A$
and $B$.

\medskip \noindent
ADD($A$, $B$)
\begin{algorithmic}
\STATE $r \la 0$
\FOR {$i \la length[A]$ to $1$}
	\STATE $C[i+1] \la \mathrm{XOR}(A[i], B[i])$
	\IF {$A[i] + B[i] + r > 1$}
		\STATE $r \la 1$
	\ELSE
		\STATE $r \la 0$
	\ENDIF
\ENDFOR
\STATE $C[1] \la r$
\end{algorithmic}

And the \textrm{XOR} function is defined as follow:\\
XOR($m$, $n$)
\begin{algorithmic}
\IF {$m = n$}
	\STATE $r \la 0$
\ELSE
	\STATE $r \la 1$
\ENDIF
\RETURN r
\end{algorithmic}
\end{document}
