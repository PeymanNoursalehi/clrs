 \documentclass[a4paper,12pt]{article}
\usepackage{algorithmic}
\newcommand{\newpar}[1]
{\bigskip \noindent \textbf{Exercises #1} \newline}
\newcommand{\newprob}[1]
{\bigskip \noindent \textbf{Problem #1} \newline}
\newcommand{\subpar}[1]
{\medskip \noindent #1.}
\newcommand{\la}{\leftarrow}
\newcommand{\ra}{\rightarrow}
\begin{document}

\newpar{2.3-2}
MERGE($A$, $p$, $q$, $r$)
\begin{algorithmic}
\STATE $n_1 \la q - p + 1$
\STATE $n_2 \la r - q$
\FOR {$i \la 1$ \textbf{to} $n_1$}
	\STATE $L[i] \la A[p + i -1]$
\ENDFOR
\FOR {$j \la 1$ \textbf{to} $n_2$}
	\STATE $R[j] \la A[q + j]$
\ENDFOR
\STATE $i \la j \la 1$
\STATE $T \la \textbf{NIL}$
\STATE $k \la p$
\WHILE {$k \le r$ \textbf{and} $T = \textbf{NIL}$}
	\IF {$i > n_1$}
		\STATE $T = R[j\ \ n_2]$
	\ELSIF {$j > n_2$}
		\STATE $T = L[i\ \ n_1]$
	\ELSIF {$L[i] \le R[j]$}
		\STATE $A[k] \la L[i]$
		\STATE $i \la i + 1$
		\STATE $k \la k + 1$
	\ELSE
		\STATE $A[k] \la R[j]$
		\STATE $j \la j + 1$
		\STATE $k \la k + 1$
	\ENDIF
\ENDWHILE
\FOR {$i \la 1$ \textbf{to} $length[T]$}
	\STATE $A[k] \la T[i]$
	\STATE $k \la k+1$
\ENDFOR
\end{algorithmic}

\newpar{2.3-3} We have $T(2)= 2 = 2 \log 2$.  Suppose that for $k \ge
1$, $T(2^k) = 2^k \lg 2^k = k 2^k$.  Thus we have, $k+1 > 1$ and
\begin{eqnarray*}
T(2^{k+1}) &=& 2T(2^k) + 2^{k+1} \\
	&=& k 2^{k+1} + 2^{k+1} \mbox{ by induction } \\
	&=& 2^{k+1} \lg 2^{k+1}
\end{eqnarray*}

\newpar{2.3-4} To insert a element in an array of $n-1$ elments, the
running time is $\Theta(n)$,  so we have the recurence
\[ T(n) = \left\{
\begin{array}{cl}
\Theta(1) & \mbox{if $n = 1$}\\
T(n-1) + \Theta(n) & \mbox{if $ n > 1$} \end{array} \right. \]	 

\newpar{2.3-5}
BINARY-SEARCH($A$, $v$) 
\begin{algorithmic}
\STATE $m \la \left\lfloor \frac{1 + length[A]}{2}\right\rfloor$
\IF {$m = 0$}
	\STATE $r \la \mathbf{NIL}$
\ELSIF {$A[m] = v$}
	\STATE $r \la m$
\ELSIF {$A[m] < v$}
	\STATE $r \la \mbox{BINARY-SEARCH}(A[1\ \ m-1],\ v)$
\ELSE
	\STATE $r \la \mbox{BINARY-SEARCH}(A[m+1\ \ length[A],\ v)$
\ENDIF
\RETURN $r$
\end{algorithmic}

Note $T(n)$ the worst case running time for an entry of size $n$.  At
each step, if we don't stop yet we run the algorithm on an entry of
size less than $\left\lfloor\frac{n+1}{2}\right\rfloor$.  Thus we have
the relation for a certain constant $C$
\[ T(n) \le C + 
T\left(\left\lfloor\frac{n+1}{2}\right\rfloor\right) \mbox{(*)}.\]
Let's show by recurrence that for $k \ge 1$
\[ T(n) \le C k + 
T\left(\left\lfloor\frac{n+2^k-1}{2^k}\right\rfloor\right).\]
Suppose we have the relation for $k$.  Using (*) we have,
\begin{eqnarray*}
T(n) &\le& C(k+1) + T\left(\left\lfloor\frac{
\left\lfloor\frac{n + 2^k - 1}{2^k}\right\rfloor + 1
}{2}\right\rfloor\right)\\ &=&
C(k+1) + T\left(\left\lfloor\frac{
\left\lfloor\frac{n + 2^k - 1}{2^k} + 1\right\rfloor
}{2}\right\rfloor\right)\\ &=&
C(k+1) + T\left(\left\lfloor\frac{
\left\lfloor\frac{n + 2^{k+1} - 1}{2^k}\right\rfloor
}{2}\right\rfloor\right)\\ &=&
C(k+1) + T\left(\left\lfloor
\frac{n + 2^{k+1} - 1}{2^{k+1}}\right\rfloor\right)
\end{eqnarray*}

\medskip
Taking $k = \lg n$, we have
\[ k \le \lg n < k+1.\]
and
\begin{eqnarray*}
\left\lfloor\frac{n + 2^k - 1}{2^k}\right\rfloor &\le&
\frac{n+2^k-1}{2^k}\\ &<&
\frac{2(2n-1)}{n}\\ &<&
4
\end{eqnarray*}
Thus,
\[ T(n) \le C \lfloor \lg n \rfloor + T(4).\]
We then deduce that,
\[ T(n) = \Theta(\lg n).\]

\newpar{2.3-6} Even if we replace the linear search through the sorted
subarray $A[1\ \ j-1]$ by a binary search , if we find the $key$ we
need to shift the array to the right to insert it.  So the running
time is always $\Theta(n^2)$.

\newpar{2.3-7}
IS-SUM($S$, $x$)
\begin{algorithmic}
\STATE $r[1] \la \mathbf{NIL}$ \COMMENT {$r$ is an array of two elements}
\STATE MERGE-SORT($S$, $1$, $length[S]$)
\STATE $i \la 1$
\WHILE {$i \le length[A]$ \textbf{and} $r[1] = \mathbf{NIL}$}
	\STATE $r[2] \la \mbox{BINARY-SEARCH}(S, v - A[i]) $
	\IF {$r[2] \not= \mathbf{NIL}$}
		\STATE $r[1] = i$
	\ENDIF
	\STATE $i \la i+1$
\ENDWHILE
\RETURN $r$	
\end{algorithmic}

We know that the running time of MERGE-SORT is $\Theta(n\lg n)$.  And
for BINARY-SEARCH, it's $\Theta(\lg n)$.  So we deduce easily that
IS-SUM is $\Theta(n\lg n)$.

\newprob{2-1} a. A list of length $k$ is sorted by insertion sort in
$\Theta(k^2)$ worst-case time. So for $\frac{n}{k}$ sublists, it's done in
$\Theta(nk)$.

\subpar{b} Given that the total length of the list is $n$,
it's easy to see that at each step of the merging the running time is
$\Theta(n)$.  Plus, the height of a binary tree who have $\frac{n}{k}$
leaves is $\lg(n/k)$.  Thus the worst-case time is
$\Theta(n \lg(n/k)$.

\subpar{c} $k$ should be $\Theta(\lg n)$.

\subpar{d} We choose $k$ such that INSERT-SORT performs
better than MERGE-SORT in input of size less than $\lg n$.

\newprob{2-2}
\subpar{a} We must prove that the array $A'$ is a permutation of the array
$A$.

\subpar{b} Consider the following property as loop invariant:
\[ A[j] \le A[k], \mbox{ for } j+1 \le k \le n.\]
\begin{itemize}
\item
\textbf{Initialization} For $j = n$, there's no $k$ verifying $j+1 \le
k \le n$ so the property holds.

\item
\textbf{Maintenance} We have $A[j-1] \le A[j]$ after executing the
body of the loop.  So finally, we have
\[ A[j-1] \le A[k], \mbox{ for } j \le k \le n.\]

\item
\textbf{Termination} At the end of the loop $j = i$.  So we have,
\[ A[i] \le A[k], \mbox{ for } i+1 \le k \le n.\]
\end{itemize}

\subpar{c} Consider the following property,
\[ A[1\ \ i-1] \mbox { is sorted and } A[i-1] \le A[k], \mbox{ for }
i \le k \le n.\]

\begin{itemize}
\item
\textbf{Initialization} For $i=1$ the array $A[1\ \ i-1]$ is empty and
$A[i-1]$ is not defined.  So the property is correct.

\item
\textbf{Maintenance} From b., after the execution of the inner loop we
have
\[A[i] \le A[k], \mbox{ for } i+1 \le k \le n.\]
And given that $A[i-1] \le A[i]$, we have the property preserved.

\item
\textbf{Termination} At the end of the loop $i = n+1$, thus the array
$A[1\ \ n]$ is sorted.
\end{itemize}

\subpar{d} The inner loop running time is $\Theta(n-i)$.  So we can
easily see that the worst-case running time of bubblesort is
$\Theta(n^2)$.

Bubble sort and insertion sort have the same worst case running time.
But in the case of bubble sort, even if the array is already sorted,
the inner loop still scan all elments of the list.  So the
running time of bubble sort on a sorted array is still $\Theta(n^2)$,
compared to $\Theta(n)$ for insertion sort.

\newprob{2-3}
\subpar{a} The running time of the code is $\Theta(n)$.

\subpar{b}  The coefficients of the polynomial is in the array $A$ of
size $n$.

\noindent
POL($x$, $A$)
\begin{algorithmic}[1]
\STATE $y \la 0$
\FOR {$ i \la 1$ \textbf{to} $length[A]$}
	\STATE $prod \la 1$
	\FOR {$j \la 1$ \textbf{to} $i-1$}
		\STATE $prod \la prod * x$
	\ENDFOR
	\STATE $y \la y + prod$
\ENDFOR
\RETURN $y$		
\end{algorithmic}

The \textbf{for} loop of lines $4$-$5$ runs in $\Theta(i-1)$.  So the
running time of the algorithm is $\Theta(n^2)$.  So it's worse than
Horner's rule.

\subpar{c} 
\begin{itemize}
\item
\textbf{Initialization}  If $i=0$, $\sum_{k=0}^{n-(i+1)}a_{k+i+1}x^k =
0$ so the property holds.

\item
\textbf{Maintenance} After executing $y \la a_i + x * y$, we have
\[ y = a_i + \sum_{k=1}^{n-i}a_{k+i}x^k =
\sum_{k=0}^{n-i}a_{k+i}x^k.\]
So after we execute $i \la i-1$, we have 
\[ y = \sum_{k=0}^{n-(i+1)}a_{k+i+1}x^k.\]

\item
\textbf{Termination} Afetr executing the \textbf{for} loop, we have $i
= -1$, thus
\[ y = \sum_{k=0}^n a_k x^k.\]
\end{itemize}

\subpar{d} The algorithm terminates because $i$ is decremented in
each iteration so after $n$ iterations, $i = -1$.  And the result is
the value of the polynom characterized by the coefficients 
$a_0, a_1, \ldots, a_n$ on the value $x$.

\newprob{2-4}
\subpar{a} $(2, 1),\ (3, 1),\ (8, 1),\ (6, 1),\ (8, 6)$.

\subpar{b} It's the array $A$ such that,
\[ A[i] = n-i, \mbox{ for } 1\le i\le n.\]
It has $\sum_{i=1}^{n-1}(n-i) = \frac{n(n-1)}{2}$ inversions.

\subpar{c} Note $I$ the number of inversions in the input array.

\noindent
INSERTION-SORT($A$)
\begin{algorithmic}[1]
\FOR {$j \la 2$ \textbf{to} $length[A]$}
	\STATE $key \la A[j]$
	\STATE $i \la j-1$
	\WHILE {$i > 0$ \textbf{and} $A[i] > key$}
		\STATE $A[i+1] \la A[i]$
		\STATE $i \la i-1$
	\ENDWHILE
	\STATE $A[i+1] \la key$
\ENDFOR
\end{algorithmic}
Let's show that the running thime of insertion sort is
$\Theta(I+n)$.  Note $A_0$ the input array before we do any
operation on $A$.

It's easy to see that: $key = A_0[j]$.  And the running time of the
\textbf{while} loop at lines $4$-$7$ is
\[ \Theta\left(\sum_{i=1\atop A[i] > key}^{j-1} 1\right).\]
Note that the array $A[1\ \ j-1]$ is a permutation of
$A_0[1\ \ j-1]$.  So there exists a permutation $\sigma$ of $1, 2,
  \dots, j-1$ such that
\[ A[i] = A_0[\sigma(i)] \mbox{ for } 1 \le i \le j-1.\]
Putting all of this together, we have
\begin{eqnarray*}
\sum_{i=1\atop A[i] > key}^{j-1} 1 &=& 
\sum_{i=1\atop A_0[\sigma(i)] > key}^{j-1} 1\\ &=&
\sum_{i=1\atop A_0[i] > key}^{j-1} 1\\ &=&
\sum_{i=1\atop A_0[i] > A_0[j]}^{j-1} 1
\end{eqnarray*}
Thus the running time of the body of the \textbf{for} loop is
\[\Theta\left(\sum_{i=1\atop A_0[i] > A_0[j]}^{j-1}1\right)
+ \Theta(1).\]
We then deduce the running time of insertion sort as
\begin{eqnarray*}
\sum_{j=2}^n \Theta\left(\sum_{i=1\atop A_0[i] > A_0[j]}^{j-1}1\right)
+ \Theta(n) &=& \Theta\left(
\sum_{j=2}^n\sum_{i=1\atop A_0[i] > A_0[j]}^{j-1}1\right) +
\Theta(n)\\&=&
\Theta(I) + \Theta(n) \\&=&
\Theta(I+n)
\end{eqnarray*}

\subpar{d}  Let's modify MERGE-SORT to count the number of inversions in $A$.
\noindent
MERGE-SORT($A$, $p$, $r$)
\begin{algorithmic}[1]
\IF {$p < r$}
	\STATE $q \la \lfloor(p+r)/2\rfloor$
	\STATE $count \la \mbox{MERGE-SORT}(A, p, q) +
		\mbox{MERGE-SORT}(A, q+1, r)$
	\STATE $count \la count + \mbox{MERGE}(A, p, q,r)$
\ELSE
	\STATE $count \la 0$
\ENDIF
\RETURN $count$
\end{algorithmic}

Here's the new version of MERGE,

\noindent
MERGE($A$, $p$, $q$, $r$)
\begin{algorithmic}[1]
\STATE $n_1 \la q - p + 1$
\STATE $n_2 \la r - q$
\STATE $count \la 0$
\STATE create arrays $L[1\ \ n_1+1]$ and $R[1\ \ n_2+1]$
\FOR {$i \la 1$ \textbf{to} $n_1$}
	\STATE $L[i] \la A[p + i -1]$
\ENDFOR
\FOR {$j \la 1$ \textbf{to} $n_2$}
	\STATE $R[j] \la A[q + j]$
\ENDFOR
\STATE $L[n_1 + 1] \la \infty$
\STATE $R[n_2 + 1] \la \infty$
\STATE $i \la 1$
\STATE $j \la 1$
\FOR {$k \la p$ \textbf{to} $r$}
	\IF {$L[i] \le R[j]$}
		\STATE $A[k] \la L[i]$
		\STATE $i \la i+1$
	\ELSE
		\STATE $count \la count + 1$
		\STATE $A[k] \la R[j]$
		\STATE $j \la j+1$
	\ENDIF
\ENDFOR
\RETURN $count$
\end{algorithmic}
Let's adopt the following notation for a property $P$,
\[ [P] = \left\{
\begin{array}{cl}
0 & \mbox{if $P$ is true}\\
1 & \mbox{if $P$ is false} \end{array} \right. \]

Again, let's note $A'$ the array $A$ before any operation in MERGE.
We then deduce easily that at the end of the \textbf{for} loop at
lines $15$-$24$, we have
\begin{eqnarray*}
count &=& \sum_{j=1}^{r-q}\sum_{i=1}^{q-p+1}[L[i] > R[j]]\\
&=& \sum_{j=q+1}^r \sum_{i=p}^q [A'[i] > A'[j]]\mbox{ (*)}
\end{eqnarray*}
We can use as loop invariant the following expression to demonstrate
this,
\[ count = \sum_{l=1}^{j-1} \sum_{k=1}^{i-1}[L[k] > R[l]].\]
Now let's take a look at the value returned by MERGE-SORT.  If $p =
r$, $count = 0$ so it's really the number of inversions in $A$.  Now
suppose that MERGE-SORT returns the number of inversions in an array
of length less than $r - p + 1$.

\medskip
Now suppose that $p < r$ and note $A_0$ the unsorted array.  By
induction after we execute line $3$, we have
\[ count = \sum_{j=q+1}^r\sum_{i=q+1}^{j-1}[A_0[i] > A_0[j]] +
\sum_{j=p}^q \sum_{i=p}^{j-1} [A_0[i] > A_0[j]].\]
Note that the array $A$ after line $3$ becomes the array $A'$ in (*).
So finally after executing line $4$ we have

\newpage
\begin{eqnarray*}
count &=& \sum_{j=q+1}^r\sum_{i=q+1}^{j-1}[A_0[i] > A_0[j]] +
\sum_{j=p}^q \sum_{i=p}^{j-1} [A_0[i] > A_0[j]] +\\&&
\sum_{j=q+1}^r \sum_{i=p}^q [A'[i] > A'[j]].
\end{eqnarray*}

But the array $A'[p\ \ q]$ is a permutation of $A_0[p\ \ q]$ and
$A'[q+1\ \ r]$ is a permutation of $A_0[q+1\ \ r]$.  So we can deduce
that 
\[ \sum_{j=q+1}^r \sum_{j=p}^q[A'[i] > A'[j]] =
\sum_{j=q+1}^r \sum_{j=p}^q[A_0[i] > A_0[j]].\]
So finally we have,
\begin{eqnarray*}
count &=& \sum_{j=q+1}^r\sum_{i=q+1}^{j-1}[A_0[i] > A_0[j]] +
\sum_{j=p}^q \sum_{i=p}^{j-1} [A_0[i] > A_0[j]] +\\&&
\sum_{j=q+1}^r \sum_{i=p}^q [A_0[i] > A_0[j]]\\&=&
\sum_{j=q+1}^r \sum_{j=p}^{j-1}[A_0[i] > A_0[j]] +
\sum_{j=p}^q \sum_{i=p}^{j-1}[A_0[i] > A_0[j]]\\&=&
\sum_{j=p}^r\sum_{i=p}^{j-1}[A_0[i] > A_0[j]]
\end{eqnarray*}

So count returns the number of inversions in $A_0$.
\end{document}
