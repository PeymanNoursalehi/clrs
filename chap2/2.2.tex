\documentclass[a4paper,12pt]{article}
\usepackage{algorithmic}
\newcommand{\newpar}[1]
{\bigskip \noindent \textbf{Exercises #1} \newline}
\newcommand{\newprob}[1]
{\bigskip \noindent \textbf{Problem #1} \newline}
\newcommand{\la}{\leftarrow}
\newcommand{\ra}{\rightarrow}
\begin{document}

\newpar{2.2-1}
We have
\[ \frac{n^3}{1000} - 100 n^2 - 100 n + 3 = \Theta(n^3).\]

\newpar{2.2-2}
SELECTION-SORT($A$)
\begin{algorithmic}
\FOR {$i \la 1$ to $length[A]$}
	\STATE $i_{min} \la \mathrm{MIN}(A, i)$
	\STATE $tmp \la A[i_{min}]$
	\STATE $A[i_{min}] \la A[i]$
	\STATE $A[i] \la tmp$
\ENDFOR
\end{algorithmic}

\smallskip \noindent
MIN($A$, $i$)
\begin{algorithmic}
\STATE $res \la i$
\FOR {$j \la i+1$ to $length[A]$}
	\IF {$A[j] < A[res]$}
		\STATE $res \la j$
	\ENDIF
\ENDFOR
\RETURN $res$
\end{algorithmic}

MIN($A$, $i$) returns the index of the minimum of the array
$A[i\ \ n]$.

Let's consider as loop invariant the property: The array $A[1\ \ i-1]$
is sorted and $ A[i-1] \le A[k], i\le k\le n$.
\begin{itemize}
\item
\textbf{Initialization: }For $i=1$, the array $A[1\ \ i-1]$ is empty
and the element $A[i-1]$ doesn't exist so the property is verified.

\item
\textbf{Maintenace: }Suppose we have the property for $i-1$.  We have,
\[ A[i_{min}] \le A[k] \mbox{ for } i\le k\le n.\]
The body of the loop swaps the value of $A[i]$ and $A[i_{min}]$. So we
have after executing the body,
\[ A[i] \le A[k] \mbox{ for } i+1\le k\le n.\]
Plus, we have $A[i-1] \le A[i]$.  So we have the property maintained.

\item
\textbf{Termination: } At the end of the loop the indice is $i$ is
$n+1$ so we have the result.
\end{itemize}

From the loop invariant for $i=n-1$, we have the array $A[1\ \ n-1]$
sorted and $A[n-1] \le A[n]$.  So the algorithm needs only to run for
the first n-1 elements.

The algorithm running time is the same for all entries of the same
size.  The function \textrm{MIN} is $\Theta(n-i)$.  We then deduce
that,
\[ T(n) = \sum_{i=1}^n\Theta(n-i).\]
We then deduce that, $T(n) = \Theta(n^2)$.

\newpar{2.2-3} If we note $n$ the size of the array.  If the element
$v$ is at the index $i$, we need to check $i$ elments of the array.  So
the number of elements of the input need to be checked on the average
is,
\begin{eqnarray*}
A(n) &=& \sum_{i=1}^n \frac{i}{n}\\
&=&\frac{n+1}{2}\\
\end{eqnarray*}
In the worst case, we need to check all the elements so $T(n) = n$.
We then deduce that $A(n) = \Theta(n)$ and $T(n) = \Theta(n)$.

\newpar{2.2-4}
We check for a special case that we know the value,  we then return this answer.
\end{document}
